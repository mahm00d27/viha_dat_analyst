% Options for packages loaded elsewhere
\PassOptionsToPackage{unicode}{hyperref}
\PassOptionsToPackage{hyphens}{url}
%
\documentclass[
]{article}
\usepackage{amsmath,amssymb}
\usepackage{lmodern}
\usepackage{ifxetex,ifluatex}
\ifnum 0\ifxetex 1\fi\ifluatex 1\fi=0 % if pdftex
  \usepackage[T1]{fontenc}
  \usepackage[utf8]{inputenc}
  \usepackage{textcomp} % provide euro and other symbols
\else % if luatex or xetex
  \usepackage{unicode-math}
  \defaultfontfeatures{Scale=MatchLowercase}
  \defaultfontfeatures[\rmfamily]{Ligatures=TeX,Scale=1}
\fi
% Use upquote if available, for straight quotes in verbatim environments
\IfFileExists{upquote.sty}{\usepackage{upquote}}{}
\IfFileExists{microtype.sty}{% use microtype if available
  \usepackage[]{microtype}
  \UseMicrotypeSet[protrusion]{basicmath} % disable protrusion for tt fonts
}{}
\makeatletter
\@ifundefined{KOMAClassName}{% if non-KOMA class
  \IfFileExists{parskip.sty}{%
    \usepackage{parskip}
  }{% else
    \setlength{\parindent}{0pt}
    \setlength{\parskip}{6pt plus 2pt minus 1pt}}
}{% if KOMA class
  \KOMAoptions{parskip=half}}
\makeatother
\usepackage{xcolor}
\IfFileExists{xurl.sty}{\usepackage{xurl}}{} % add URL line breaks if available
\IfFileExists{bookmark.sty}{\usepackage{bookmark}}{\usepackage{hyperref}}
\hypersetup{
  pdftitle={Data Analyst Assignment},
  pdfauthor={Mahmoodur Rahman},
  hidelinks,
  pdfcreator={LaTeX via pandoc}}
\urlstyle{same} % disable monospaced font for URLs
\usepackage[margin=1in]{geometry}
\usepackage{color}
\usepackage{fancyvrb}
\newcommand{\VerbBar}{|}
\newcommand{\VERB}{\Verb[commandchars=\\\{\}]}
\DefineVerbatimEnvironment{Highlighting}{Verbatim}{commandchars=\\\{\}}
% Add ',fontsize=\small' for more characters per line
\usepackage{framed}
\definecolor{shadecolor}{RGB}{248,248,248}
\newenvironment{Shaded}{\begin{snugshade}}{\end{snugshade}}
\newcommand{\AlertTok}[1]{\textcolor[rgb]{0.94,0.16,0.16}{#1}}
\newcommand{\AnnotationTok}[1]{\textcolor[rgb]{0.56,0.35,0.01}{\textbf{\textit{#1}}}}
\newcommand{\AttributeTok}[1]{\textcolor[rgb]{0.77,0.63,0.00}{#1}}
\newcommand{\BaseNTok}[1]{\textcolor[rgb]{0.00,0.00,0.81}{#1}}
\newcommand{\BuiltInTok}[1]{#1}
\newcommand{\CharTok}[1]{\textcolor[rgb]{0.31,0.60,0.02}{#1}}
\newcommand{\CommentTok}[1]{\textcolor[rgb]{0.56,0.35,0.01}{\textit{#1}}}
\newcommand{\CommentVarTok}[1]{\textcolor[rgb]{0.56,0.35,0.01}{\textbf{\textit{#1}}}}
\newcommand{\ConstantTok}[1]{\textcolor[rgb]{0.00,0.00,0.00}{#1}}
\newcommand{\ControlFlowTok}[1]{\textcolor[rgb]{0.13,0.29,0.53}{\textbf{#1}}}
\newcommand{\DataTypeTok}[1]{\textcolor[rgb]{0.13,0.29,0.53}{#1}}
\newcommand{\DecValTok}[1]{\textcolor[rgb]{0.00,0.00,0.81}{#1}}
\newcommand{\DocumentationTok}[1]{\textcolor[rgb]{0.56,0.35,0.01}{\textbf{\textit{#1}}}}
\newcommand{\ErrorTok}[1]{\textcolor[rgb]{0.64,0.00,0.00}{\textbf{#1}}}
\newcommand{\ExtensionTok}[1]{#1}
\newcommand{\FloatTok}[1]{\textcolor[rgb]{0.00,0.00,0.81}{#1}}
\newcommand{\FunctionTok}[1]{\textcolor[rgb]{0.00,0.00,0.00}{#1}}
\newcommand{\ImportTok}[1]{#1}
\newcommand{\InformationTok}[1]{\textcolor[rgb]{0.56,0.35,0.01}{\textbf{\textit{#1}}}}
\newcommand{\KeywordTok}[1]{\textcolor[rgb]{0.13,0.29,0.53}{\textbf{#1}}}
\newcommand{\NormalTok}[1]{#1}
\newcommand{\OperatorTok}[1]{\textcolor[rgb]{0.81,0.36,0.00}{\textbf{#1}}}
\newcommand{\OtherTok}[1]{\textcolor[rgb]{0.56,0.35,0.01}{#1}}
\newcommand{\PreprocessorTok}[1]{\textcolor[rgb]{0.56,0.35,0.01}{\textit{#1}}}
\newcommand{\RegionMarkerTok}[1]{#1}
\newcommand{\SpecialCharTok}[1]{\textcolor[rgb]{0.00,0.00,0.00}{#1}}
\newcommand{\SpecialStringTok}[1]{\textcolor[rgb]{0.31,0.60,0.02}{#1}}
\newcommand{\StringTok}[1]{\textcolor[rgb]{0.31,0.60,0.02}{#1}}
\newcommand{\VariableTok}[1]{\textcolor[rgb]{0.00,0.00,0.00}{#1}}
\newcommand{\VerbatimStringTok}[1]{\textcolor[rgb]{0.31,0.60,0.02}{#1}}
\newcommand{\WarningTok}[1]{\textcolor[rgb]{0.56,0.35,0.01}{\textbf{\textit{#1}}}}
\usepackage{graphicx}
\makeatletter
\def\maxwidth{\ifdim\Gin@nat@width>\linewidth\linewidth\else\Gin@nat@width\fi}
\def\maxheight{\ifdim\Gin@nat@height>\textheight\textheight\else\Gin@nat@height\fi}
\makeatother
% Scale images if necessary, so that they will not overflow the page
% margins by default, and it is still possible to overwrite the defaults
% using explicit options in \includegraphics[width, height, ...]{}
\setkeys{Gin}{width=\maxwidth,height=\maxheight,keepaspectratio}
% Set default figure placement to htbp
\makeatletter
\def\fps@figure{htbp}
\makeatother
\setlength{\emergencystretch}{3em} % prevent overfull lines
\providecommand{\tightlist}{%
  \setlength{\itemsep}{0pt}\setlength{\parskip}{0pt}}
\setcounter{secnumdepth}{-\maxdimen} % remove section numbering
\usepackage{booktabs}
\usepackage{longtable}
\usepackage{array}
\usepackage{multirow}
\usepackage{wrapfig}
\usepackage{float}
\usepackage{colortbl}
\usepackage{pdflscape}
\usepackage{tabu}
\usepackage{threeparttable}
\usepackage{threeparttablex}
\usepackage[normalem]{ulem}
\usepackage{makecell}
\usepackage{xcolor}
\ifluatex
  \usepackage{selnolig}  % disable illegal ligatures
\fi

\title{Data Analyst Assignment}
\author{Mahmoodur Rahman}
\date{09 June, 2023}

\begin{document}
\maketitle

\hypertarget{assignment-overview}{%
\subsection{Assignment Overview}\label{assignment-overview}}

In this assignment, you will connect to an SQLite database containing
information about patients tested for MRSA, CDI, and COVID-19. You will
run some analytics on the data and develop insights into the patient
distribution, infection rates, and treatment effectiveness.

NOTE: Please remember to load in any packages you plan to use for your
analysis.

\hypertarget{task-1-connect-to-the-database}{%
\subsection{Task 1: Connect to the
Database}\label{task-1-connect-to-the-database}}

Your first task is to connect to the SQLite database infection.db using
the RSQLite package. Once you've established a connection, list the
tables in the database. You can find the SQL file located in the same
zip file as this assignment.

\begin{table}[!h]

\caption{\label{tab:task ONE}List of tables}
\centering
\begin{tabular}[t]{l}
\hline
Name of Dataset\\
\hline
CDI\\
\hline
COVID19\\
\hline
MRSA\\
\hline
\end{tabular}
\end{table}

\hypertarget{task-2-load-the-data-into-data-frames}{%
\subsection{Task 2: Load the Data into Data
Frames}\label{task-2-load-the-data-into-data-frames}}

Next, write a query to select all data from each of the three tables:
MRSA, CDI, and COVID19. Store the result of each query in a separate
data frame.

\begin{Shaded}
\begin{Highlighting}[]
\CommentTok{\# saving tables according to names followed by "\_df"}
\ControlFlowTok{for}\NormalTok{ (df.name }\ControlFlowTok{in}\NormalTok{ DBI}\SpecialCharTok{::}\FunctionTok{dbListTables}\NormalTok{(con)) \{}
  \FunctionTok{assign}\NormalTok{(}\FunctionTok{paste0}\NormalTok{(df.name, }\StringTok{"\_df"}\NormalTok{),}
\NormalTok{         DBI}\SpecialCharTok{::}\FunctionTok{dbReadTable}\NormalTok{(con,df.name))}
\NormalTok{\}}
\end{Highlighting}
\end{Shaded}

Variable Description:

\begin{itemize}
\tightlist
\item
  patient\_identifier: the unique patient id
\item
  age: current age in years
\item
  unit: the name of unit where patient was staying at when got tested,
  format: ``floor level - facility name'' (There are 10 different
  facilities in the tables)
\item
  room: the room number where patient was staying at when got tested
\item
  bed: the bed code where patient was staying at when got tested
\item
  result: the test result: Positive or Negative
\item
  treatment: the medicines patient was treated
\end{itemize}

\hypertarget{task-3-analyze-the-data}{%
\subsection{Task 3: Analyze the Data}\label{task-3-analyze-the-data}}

Now, perform some basic analysis on each data frame. This could include
(but not limited to):

\begin{enumerate}
\def\labelenumi{\arabic{enumi}.}
\tightlist
\item
  Count the number of patients in each facility.
\item
  Identify the unit with the highest number of positive results in each
  facility.
\item
  Calculate the infection rate for each disease (infection rate = number
  of positive results/total number of results).
\end{enumerate}

Feel free to be creative during this section and come up with some
unique ways to interpret and analyze the data.

\begin{table}[!h]

\caption{\label{tab:unnamed-chunk-1}Number of patients in each facility}
\centering
\begin{tabular}[t]{l|>{\raggedleft\arraybackslash}p{1in}|>{\raggedleft\arraybackslash}p{1in}|>{\raggedleft\arraybackslash}p{1in}}
\hline
Facility Name & Number of patients in CDI dataframe & Number of patients in Covid-19 dataframe & Number of patients in MRSA dataframe\\
\hline
Blossom Community Hospital & 549 & 435 & 175\\
\hline
East Valley Medical Clinic & 250 & 1339 & 201\\
\hline
Featherfall Medical Center & 890 & 697 & 798\\
\hline
Goldvalley Medical Clinic & 605 & 560 & 569\\
\hline
Hyacinth General Hospital & 1526 & 998 & 416\\
\hline
Kindred Medical Clinic & 583 & 318 & 125\\
\hline
Magnolia Hospital & 763 & 1883 & 613\\
\hline
Maple Valley Medical Center & 535 & 633 & 364\\
\hline
Rosewood Hospital Center & 988 & 1283 & 435\\
\hline
Ruby Valley Hospital & 377 & 342 & 148\\
\hline
Summit Community Hospital & 508 & 777 & 503\\
\hline
Wildflower Medical Center & 426 & 735 & 653\\
\hline
\end{tabular}
\end{table}

\begin{table}[!h]

\caption{\label{tab:unnamed-chunk-1}Units with the highest number of positive results in each facility}
\centering
\begin{tabular}[t]{l|>{\raggedright\arraybackslash}p{0.5in}|>{\raggedleft\arraybackslash}p{0.5in}|>{\raggedright\arraybackslash}p{0.5in}|>{\raggedleft\arraybackslash}p{0.5in}|>{\raggedright\arraybackslash}p{0.5in}|>{\raggedleft\arraybackslash}p{0.5in}|>{}p{0.5in}|>{}p{0.5in}}
\hline
\multicolumn{1}{c|}{Facility name} & \multicolumn{2}{c|}{CDI Data} & \multicolumn{2}{c|}{Covid-19 Data} & \multicolumn{2}{c}{MRSA Data} \\
\cline{1-1} \cline{2-3} \cline{4-5} \cline{6-7}
Facility name & Units with most positive patients & Number of patients & Units with most positive patients & Number of patients & Units with most positive patients & Number of patients\\
\hline
Summit Community Hospital & Floor 1 & 38 & Floor 5 & 51 & Floor 1 & 46\\
\hline
East Valley Medical Clinic & Floor 2 & 19 & Floor 4 & 168 & Floor 3 & 24\\
\hline
Kindred Medical Clinic & Floor 2 & 35 & Floor 1 & 25 & Floor 1 & 15\\
\hline
Maple Valley Medical Center & Floor 2 & 35 & Floor 1 & 50 & Floor 1 & 29\\
\hline
Ruby Valley Hospital & Floor 2 & 39 & Floor 2 & 30 & Floor 4 & 27\\
\hline
Ruby Valley Hospital & Floor 2 & 39 & Floor 4 & 30 & Floor 4 & 27\\
\hline
Blossom Community Hospital & Floor 4 & 26 & Floor 5 & 43 & Floor 4 & 23\\
\hline
Magnolia Hospital & Floor 4 & 45 & Floor 2 & 101 & Floor 2 & 92\\
\hline
Rosewood Hospital Center & Floor 4 & 67 & Floor 5 & 66 & Floor 2 & 43\\
\hline
Wildflower Medical Center & Floor 4 & 19 & Floor 3 & 61 & Floor 4 & 115\\
\hline
Featherfall Medical Center & Floor 5 & 71 & Floor 3 & 55 & Floor 3 & 131\\
\hline
Goldvalley Medical Clinic & Floor 5 & 38 & Floor 2 & 32 & Floor 1 & 56\\
\hline
Hyacinth General Hospital & Floor 5 & 161 & Floor 4 & 53 & Floor 1 & 46\\
\hline
\end{tabular}
\end{table}

\begin{table}[!h]

\caption{\label{tab:unnamed-chunk-1}Infection rate for each diseas}
\centering
\begin{tabular}[t]{l|r|r|r}
\hline
Diseases & Total Number of Tests & Number of Positive Tests & Infection Rate\\
\hline
CDI & 8000 & 1222 & 0.15\\
\hline
Covid-19 & 10000 & 1721 & 0.17\\
\hline
MRSA & 5000 & 1514 & 0.30\\
\hline
\end{tabular}
\end{table}

\hypertarget{task-4-develop-insights}{%
\subsection{Task 4: Develop Insights}\label{task-4-develop-insights}}

Based on your analysis and visualization, develop some insights into the
data. Discuss any patterns or trends you observed, any surprising
results, and any limitations or potential improvements to your analysis.

\begin{Shaded}
\begin{Highlighting}[]
\CommentTok{\# Number of patients attended largely varies by facility and by disease itself}

\CommentTok{\# In almost all facilities, the number of Covid{-}19 patients are more in number, }
\CommentTok{\# compared to other two diseases}

\CommentTok{\# Among the two diseases primarily caused by antimicrobial resistance, }
\CommentTok{\# number of CDI patients are more in number}

\CommentTok{\# In case of number of test{-}positive patients in a facility, }
\CommentTok{\# we can see that it differs largely between diseases}

\CommentTok{\# We have a tie in number of positive patients in unit floor 2 and }
\CommentTok{\# floor 4 in Ruby Valley Hospital in case of Covid{-}19 positive patients}

\CommentTok{\# Interestingly MRSA patients are the least in numbers attended by the facilities, }
\CommentTok{\# but the disease has the hoghest Infection rate compared to other two diseases.}
\end{Highlighting}
\end{Shaded}

\newpage

\hypertarget{submission-methods}{%
\subsection{Submission Methods}\label{submission-methods}}

You can submit your assignment as a RMarkdown file, PDF, HTML, or Word
document.

To generate a PDF, HTML, or Word document from your RMarkdown file,
click on the ``Knit'' button in the RStudio toolbar and select the
output format you want.

If you choose to submit PDF, HTML, or Word document, please make sure we
can see both the results and your code(with echo = TRUE in your
RMarkdown chunks). This is crucial for our review process.

\hypertarget{alternative-submission-methods-and-programming-languages}{%
\subsubsection{Alternative Submission Methods and Programming
Languages}\label{alternative-submission-methods-and-programming-languages}}

We acknowledge the diverse skills of our participants, so you are
welcome to complete this assignment using the programming language of
your choice. The provided RMarkdown file is simply a guide. You may
choose to use Python, Julia, SQL, SAS, or any other language that you
are comfortable with and that can effectively interact with the SQLite
database.

The medium of your assignment submission can also vary. If you choose
not to use the provided RMarkdown file, please ensure that your chosen
medium is clear, well-organized, and accessible for the selection
committee. Some examples of alternative submissions include:

Hosting a temporary website: Create a website to present your
assignment. The website should clearly display your code, output,
visualizations, and insights. Ensure that the website is live and
accessible by the selection committee until the review process is
complete.

Sharing a GitHub repo: You can push your code, along with any generated
outputs and visualizations, to a public GitHub repository. Make sure to
include a README file explaining the structure of your repo and how to
run your code.

Sharing a Kaggle notebook: You can develop your assignment in a Kaggle
notebook. Make sure the notebook is public and can be accessed by the
selection committee.

Please include the link to your website, GitHub repo, Kaggle notebook,
or any other chosen medium when you submit your assignment.

\end{document}
